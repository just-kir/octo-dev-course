\documentclass[12pt]{article}

\usepackage{listings}
\lstloadlanguages{Ruby}
%\lstset{%
%basicstyle=\ttfamily\color{black},
%commentstyle = \ttfamily\color{red},
%keywordstyle=\ttfamily\color{blue},
%stringstyle=\color{orange}}
%\usepackage[dvipsnames]{xcolor}

\usepackage[utf8]{inputenc}
\usepackage[russian]{babel}
\usepackage{amsmath}
\usepackage{amssymb}
\usepackage{geometry}
\usepackage{graphicx}
\usepackage{braket}
\usepackage{hyperref}
\usepackage{esint}
\geometry{top=2cm} %поле сверху
\geometry{bottom=2cm} %поле снизу
\geometry{left=2cm} %поле справа
\geometry{right=2cm}
\usepackage{graphicx}

%\newcommand{\ud}{\mathop{\mathrm{{}d}}\mathopen{}}

\itemsep=0pt

\begin{document}
\begin{enumerate}
\item Раскодируйте сообщение, которое закодировано с помощью изученного на занятии кода Шеннона-Фано: 1111100001101111100100110111101.
\item Постройте дерево Хаффмана для фразы <<Мама мыла ламу>>. Найдите коды всех входящих в нее символов и закодируйте сообщение. Чему равен коэффициент сжатия в сравнении с равномерным кодом минимальной длины? С однобайтовой кодировкой?
\end{enumerate}

\begin{enumerate}
\item Раскодируйте сообщение, которое закодировано с помощью изученного на занятии кода Шеннона-Фано: 1111100001101111100100110111101.
\item Постройте дерево Хаффмана для фразы <<Мама мыла ламу>>. Найдите коды всех входящих в нее символов и закодируйте сообщение. Чему равен коэффициент сжатия в сравнении с равномерным кодом минимальной длины? С однобайтовой кодировкой?
\end{enumerate}

\begin{enumerate}
\item Раскодируйте сообщение, которое закодировано с помощью изученного на занятии кода Шеннона-Фано: 1111100001101111100100110111101.
\item Постройте дерево Хаффмана для фразы <<Мама мыла ламу>>. Найдите коды всех входящих в нее символов и закодируйте сообщение. Чему равен коэффициент сжатия в сравнении с равномерным кодом минимальной длины? С однобайтовой кодировкой?
\end{enumerate}

\begin{enumerate}
\item Раскодируйте сообщение, которое закодировано с помощью изученного на занятии кода Шеннона-Фано: 1111100001101111100100110111101.
\item Постройте дерево Хаффмана для фразы <<Мама мыла ламу>>. Найдите коды всех входящих в нее символов и закодируйте сообщение. Чему равен коэффициент сжатия в сравнении с равномерным кодом минимальной длины? С однобайтовой кодировкой?
\end{enumerate}

\begin{enumerate}
\item Раскодируйте сообщение, которое закодировано с помощью изученного на занятии кода Шеннона-Фано: 1111100001101111100100110111101.
\item Постройте дерево Хаффмана для фразы <<Мама мыла ламу>>. Найдите коды всех входящих в нее символов и закодируйте сообщение. Чему равен коэффициент сжатия в сравнении с равномерным кодом минимальной длины? С однобайтовой кодировкой?
\end{enumerate}

\begin{enumerate}
\item Раскодируйте сообщение, которое закодировано с помощью изученного на занятии кода Шеннона-Фано: 1111100001101111100100110111101.
\item Постройте дерево Хаффмана для фразы <<Мама мыла ламу>>. Найдите коды всех входящих в нее символов и закодируйте сообщение. Чему равен коэффициент сжатия в сравнении с равномерным кодом минимальной длины? С однобайтовой кодировкой?
\end{enumerate}

\begin{enumerate}
\item Раскодируйте сообщение, которое закодировано с помощью изученного на занятии кода Шеннона-Фано: 1111100001101111100100110111101.
\item Постройте дерево Хаффмана для фразы <<Мама мыла ламу>>. Найдите коды всех входящих в нее символов и закодируйте сообщение. Чему равен коэффициент сжатия в сравнении с равномерным кодом минимальной длины? С однобайтовой кодировкой?
\end{enumerate}



\end{document}