%%% Поля и разметка страницы %%%
\documentclass[a4paper,12pt]{article}
\usepackage{lscape}		% Для включения альбомных страниц
\pagestyle{empty}		% Не нумеровать
\usepackage[left=2cm,right=2cm,
top=3cm,bottom=1.5cm,bindingoffset=0cm]{geometry} % поля


%%% Кодировки и шрифты %%%
\usepackage{cmap}						% Улучшенный поиск русских слов в полученном pdf-файле
\usepackage[T2A]{fontenc}				% Поддержка русских букв
\usepackage[utf8]{inputenc}				% Кодировка utf8
\usepackage[english, russian]{babel}	% Языки: русский, английский

%%% Математические пакеты %%%
\usepackage{amsthm,amsfonts,amsmath,amssymb,amscd} % Математические дополнения от AMS

%%% Оформление абзацев %%%
\usepackage{indentfirst} % Красная строка

\usepackage{enumitem}
\usepackage{verbatim}

%%% Цвета %%%
\usepackage[usenames]{color}
\usepackage{color}
\usepackage{colortbl}

%%% Таблицы %%%
\usepackage{longtable}					% Длинные таблицы
\usepackage{multirow,makecell,array}	% Улучшенное форматирование таблиц
\usepackage{multicol}

%%% Общее форматирование
\usepackage[singlelinecheck=off,center]{caption}	% Многострочные подписи
\usepackage{soul}									% Поддержка переносоустойчивых подчёркиваний и зачёркиваний
\usepackage{icomma}

%%% Библиография %%%
\usepackage{cite} % Красивые ссылки на литературу

%%% Гиперссылки %%%
\usepackage[plainpages=false,pdfpagelabels=false]{hyperref}
\definecolor{linkcolor}{rgb}{0.9,0,0}
\definecolor{citecolor}{rgb}{0,0.6,0}
\definecolor{urlcolor}{rgb}{0,0,1}
\hypersetup{
    colorlinks, linkcolor={linkcolor},
    citecolor={citecolor}, urlcolor={urlcolor}
}

%%% Изображения %%%
\usepackage{graphicx}		% Подключаем пакет работы с графикой
\graphicspath{{images/}}	% Пути к изображениям

%%% Выравнивание и переносы %%%
\sloppy					% Избавляемся от переполнений
\clubpenalty=10000		% Запрещаем разрыв страницы после первой строки абзаца
\widowpenalty=10000		% Запрещаем разрыв страницы после последней строки абзаца

%%% Библиография %%%
\makeatletter
\bibliographystyle{utf8gost705u}	% Оформляем библиографию в соответствии с ГОСТ 7.0.5
\renewcommand{\@biblabel}[1]{#1.}	% Заменяем библиографию с квадратных скобок на точку:
\makeatother

%%% Колонтитулы %%%
\let\Sectionmark\sectionmark
\def\sectionmark#1{\def\Sectionname{#1}\Sectionmark{#1}}
\makeatletter
\newcommand*{\currentname}{\@currentlabelname}
\renewcommand{\@oddhead}{\it \vbox{\hbox to \textwidth%
		{\hfil Информатика и ИКТ. 11 класс\hfil\strut}\hbox to \textwidth%
		{\today \hfil Физтех-лицей\strut}\hrule}}
\makeatother

%%%%%%%%%%%%%%%%%%%%%%%%%%%%%%%%%%%%%%%%%%%%%%%%%%%%%%%%%%%%%%%%%%%%%%%%%%%%%%%%%%%
\begin{document}

\begin{center}
\Large{\textbf{Домашнее задание на 28.11.2014}}
\end{center}
\section*{1 вариант} 

\subsection*{1}
Сколько единиц в двоичной записи числа 4095? 

\subsection*{2}
Дано: $a = D7_{16}$ и $b = 331_8$. Какое из чисел с, записанных в двоичной системе счисления, удовлетворяет  неравенству $a < c < b$?
\begin{itemize}
 \item $11011001_2$
 \item $11011100_2$
 \item $11010111_2$
 \item $11011000_2$
\end{itemize}

\subsection*{3}
Для хранения целого числа \textit{со знаком} используется один байт. Сколько единиц содержит внутреннее представление числа $(-78)$? 

\subsection*{4}
Сколько единиц в двоичной записи числа  $4^{2014} + 2^{2015} - 8$?

\subsection*{5*}
Укажите все основания систем счисления, в которых запись числа $23$ оканчивается на $2$.

\subsection*{6}
Решите уравнение $60_8 + x  = 120_7$. Ответ запишите в шестнадцатеричной системе счисления.

\newpage
\section*{2 вариант} 

\subsection*{1}
Сколько единиц в двоичной записи числа 2049? 

\subsection*{2}
Для хранения целого числа со знаком используется один байт. Сколько единиц содержит внутреннее представление числа $(-35)$?

\subsection*{3}
Переведите $24.17_{10}$ в двоичную систему счисления.

\subsection*{4}
Сколько единиц в двоичной записи числа $4^{2016}  + 2^{2018} - 8^{600} + 6$?

\subsection*{5}
Укажите через запятую в порядке возрастания все десятичные числа, не превосходящие 25, запись которых в системе счисления с основанием четыре оканчивается на 11?  

\subsection*{6*}
Найдите все основания систем счисления, в которых запись числа 31 оканчивается на 11.
\end{document}
