%%% Поля и разметка страницы %%%
\documentclass[a4paper,12pt]{article}
\usepackage{lscape}		% Для включения альбомных страниц
\pagestyle{empty}		% Не нумеровать
\usepackage[left=2cm,right=2cm,
top=3cm,bottom=1.5cm,bindingoffset=0cm]{geometry} % поля


%%% Кодировки и шрифты %%%
\usepackage{cmap}						% Улучшенный поиск русских слов в полученном pdf-файле
\usepackage[T2A]{fontenc}				% Поддержка русских букв
\usepackage[utf8]{inputenc}				% Кодировка utf8
\usepackage[english, russian]{babel}	% Языки: русский, английский



%%% Математические пакеты %%%
\usepackage{amsthm,amsfonts,amsmath,amssymb,amscd} % Математические дополнения от AMS

%%% Оформление абзацев %%%
\usepackage{indentfirst} % Красная строка

\usepackage{enumitem}
\usepackage{verbatim}
\usepackage{listings}

%%% Цвета %%%
\usepackage[usenames]{color}
\usepackage{color}
\usepackage{colortbl}

%%% Таблицы %%%
\usepackage{longtable}					% Длинные таблицы
\usepackage{multirow,makecell,array}	% Улучшенное форматирование таблиц
\usepackage{multicol}

%%% Общее форматирование
\usepackage[singlelinecheck=off,center]{caption}	% Многострочные подписи
\usepackage{soul}									% Поддержка переносоустойчивых подчёркиваний и зачёркиваний
\usepackage{icomma}

%%% Библиография %%%
\usepackage{cite} % Красивые ссылки на литературу

%%% Гиперссылки %%%
\usepackage[plainpages=false,pdfpagelabels=false]{hyperref}
\definecolor{linkcolor}{rgb}{0.9,0,0}
\definecolor{citecolor}{rgb}{0,0.6,0}
\definecolor{urlcolor}{rgb}{0,0,1}
\hypersetup{
    colorlinks, linkcolor={linkcolor},
    citecolor={citecolor}, urlcolor={urlcolor}
}

%%% Изображения %%%
\usepackage{graphicx}		% Подключаем пакет работы с графикой
\graphicspath{{images/}}	% Пути к изображениям

%%% Выравнивание и переносы %%%
\sloppy					% Избавляемся от переполнений
\clubpenalty=10000		% Запрещаем разрыв страницы после первой строки абзаца
\widowpenalty=10000		% Запрещаем разрыв страницы после последней строки абзаца

%%%Оформление блоков кода %%%
\definecolor{dkgreen}{rgb}{0,0.6,0}
\definecolor{gray}{rgb}{0.5,0.5,0.5}
\definecolor{mauve}{rgb}{0.58,0,0.82}
\lstset{frame=tb,
  language=Java,
  aboveskip=3mm,
  belowskip=3mm,
  showstringspaces=false,
  columns=flexible,
  basicstyle={\small\ttfamily},
  numbers=none,
  numberstyle=\tiny\color{gray},
  keywordstyle=\color{blue},
  commentstyle=\color{dkgreen},
  stringstyle=\color{mauve},
  breaklines=true,
  breakatwhitespace=true,
  tabsize=3
}


%%% Библиография %%%
\makeatletter
\bibliographystyle{utf8gost705u}	% Оформляем библиографию в соответствии с ГОСТ 7.0.5
\renewcommand{\@biblabel}[1]{#1.}	% Заменяем библиографию с квадратных скобок на точку:
\makeatother

%%% Колонтитулы %%%
\let\Sectionmark\sectionmark
\def\sectionmark#1{\def\Sectionname{#1}\Sectionmark{#1}}
\makeatletter
\newcommand*{\currentname}{\@currentlabelname}
\renewcommand{\@oddhead}{\it \vbox{\hbox to \textwidth%
		{\hfil Информатика и ИКТ, 11 класс\hfil\strut}\hbox to \textwidth%
		{\today \hfil Физтех-лицей\strut}\hrule}}
\makeatother




%%%%%%%%%%%%%%%%%%%%%%%%%%%%%%%%%%%%%%%%%%%%%%%%%%%%%%%%%%%%%%%%%%%%%%%%%%%%%%%%%%%
\begin{document}
\begin{center}
\Large{\textbf{Лабораторная работа по графике №2, 26 декабря}}
\end{center}
В этой работе мы продолжим учиться работать с графикой в python. Теперь не будет поясняющих примеров. Вам до всего нужно будет догадаться самостоятельно :-) Правила такие же, как и в предыдущей контрольной работе. Вам дается код программы, вы его набираете, смотрите, что получается и комментируете каждую строчку. После выполнения присылаете в { \bfseries одном файле } все упражнения на почты $dontwritehim@gmail.com$ и $i\_khisambeev@mail.ru$.

\section{Первое упражение --- гистограмма}
\begin{lstlisting}[language=python]
import matplotlib.pyplot as plt
import numpy as np
 
y = np.random.randn(1000)
 
plt.hist(y, 25)
plt.show()
\end{lstlisting}
Вам нужно догадаться, что делает функция $randn$. Это не просто случайные числа. Можете попробовать заполучить подсказку у Ильдара :-)



\section{Второе упражение: учет ошибок}
В практической науке, эксперимент даже при максимальной точности измерений всегда вносит свою погрешность. Для того, чтобы учесть это и указать возможный разброс вокруг значения, считаемого истинным, вводят error bars, которые на кривой для полученных точек показывают своеобразный доверительный интервал:
\begin{lstlisting}[language=python]
import matplotlib.pyplot as plt
import numpy as np
 
x = np.arange(0, 4, 0.2)
y = np.exp(-x)
e1 = 0.1 * np.abs(np.random.randn(len(y)))
 
plt.errorbar(x, y, yerr=e1, fmt='.-')
plt.show()
\end{lstlisting}
Особенно тщательно разберитесь с тем, как работает $np.arange$. Можно воспользоваться гуглом ;-)

\section{Третье упражнение: круговая диаграмма}
Такой вид диаграмм используется, когда для нас существенно сравнение вклада каждого участника в общее дело. Поскольку это дело похоже на разрезание пирога, то и диаграммы также называют pie charts, а функцию pie(). Первым аргументом она принимает x-последовательность из внесенных единиц.
\begin{lstlisting}[language=python]
import matplotlib.pyplot as plt
 
plt.figure(figsize=(7,5))
x = [18, 15, 11, 9, 8, 6]
labels = ['Java', 'C', 'C++', 'PHP', '(Visual) Basic', 'Python']
explode = [0, 0, 0, 0, 0, 0.2]
 
plt.pie(x, labels = labels, explode = explode, autopct = '%1.1f%%', shadow=True);
plt.show()
\end{lstlisting}
Кстати, эта статистика уже давно неверная. 

\section{Четвертое упражнение: график рассеяния}
Такой тип графиков позволяет изображать одновременно два множества данных, которые не образуют кривой, а именно двухмерное множество точек. Каждая точка имеет две координаты. График рассеяния часто используется для определения связи между двумя величинами и позволяет определить более точные пределы измерений.
\begin{lstlisting}[language=python]
import matplotlib.pyplot as plt
import numpy as np
 
x = np.random.randn(1000)
y = np.random.randn(1000)
 
size = 50*np.random.randn(1000)
colors = np.random.rand(1000)
 
plt.scatter(x, y, s=size, c=colors)
plt.show()
\end{lstlisting}

\section{Пятое упражнение: полярные координаты}
Кроме наиболее часто используемой декартовой системы координат, довольно широко применяется и полярная система координат, удобная в различных радиальных задачах, координаты точек в ней задается с помощью радиус-вектора $\rho$, идущего из начала координат и угла $\theta$. Угол может быть задан в радианах или градусах, matplotlib использует градусы.
\begin{lstlisting}[language=python]
import matplotlib.pyplot as plt
import numpy as np
 
theta = np.arange(0., 2., 1./180.)*np.pi    
 
plt.polar(3*theta, theta/5)        
plt.polar(theta, np.cos(4*theta))  
plt.polar(theta, [1.4]*len(theta))  
 
plt.show()
\end{lstlisting}
\end{document}