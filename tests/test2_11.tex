\documentclass[12pt]{article}

\usepackage{listings}
\lstloadlanguages{Ruby}
%\lstset{%
%basicstyle=\ttfamily\color{black},
%commentstyle = \ttfamily\color{red},
%keywordstyle=\ttfamily\color{blue},
%stringstyle=\color{orange}}
%\usepackage[dvipsnames]{xcolor}

\usepackage[utf8]{inputenc}
\usepackage[russian]{babel}
\usepackage{amsmath}
\usepackage{amssymb}
\usepackage{geometry}
\usepackage{graphicx}
\usepackage{braket}
\usepackage{hyperref}
\usepackage{esint}
\geometry{top=2cm} %поле сверху
\geometry{bottom=2cm} %поле снизу
\geometry{left=2cm} %поле справа
\geometry{right=2cm}
\usepackage{graphicx}

%\newcommand{\ud}{\mathop{\mathrm{{}d}}\mathopen{}}

\itemsep=0pt

\begin{document}


\section*{Задачи}
\subsection*{Теория}
\begin{enumerate}
\item Документ объёмом 10 Мбайт можно передать с одного компьютера на другой двумя способами:
	\begin{itemize}
			\vspace*{-2ex}
		    \setlength{\itemsep}{0pt}%
		    \setlength{\parskip}{0pt}%
		\item[А)] сжать архиватором, передать архив по каналу связи, распаковать; 
		\item[Б)] передать по каналу связи без использования архиватора.
	\end{itemize}
\vspace*{-1.7ex}
Какой способ быстрее и насколько, если
	\begin{itemize}
			\vspace*{-1.5ex}
			\setlength{\itemsep}{0pt}%
		    \setlength{\parskip}{0pt}%
		\item средняя скорость передачи данных по каналу связи составляет $2^{20}$ бит в секунду,
		\item объём сжатого архиватором документа равен 30\% от исходного,
		\item время, требуемое на сжатие документа, - 12 секунд, на распаковку - 2 секунды?
	\end{itemize}

\item Книжка, в которой 10 страниц текста (каждая страница содержит 32 строки по 32 символа в каждой), закодирована в 8-битной кодировке. Сколько секунд потребуется для передачи этой книжки по линии связи со скоростью 2 Кбайт в секунду?
\item Какое наименьшее число символов должно быть в алфавите, чтобы с помощью всевозможных 4-буквенных слов, состоящих из символов данного алфавита, можно было передать не менее 10 различных сообщений? 
\item В соревнованиях по ориентированию участвуют 430 спортсменов. Специальное устройство регистрирует финиш каждого из участников, записывая его номер с использованием минимально возможного количества битов, одинакового для каждого спортсмена. Каков будет информационный объём сообщения (в байтах), записанного устройством, после того как финишируют 400 спортсменов?
\end{enumerate}

\subsection*{Практика}
\begin{enumerate}
\item Вам дан текстовый файл input.txt. В нем содержится две строки. Вам требуется найти все вхождения второй строки в первую и вывести на экран их количество. Например, строка <<aba>> входит в строку <<ababbababa>> три раза.

\item Cтрока S1 называется анаграммой строки S2, если она получается из S2 перестановкой символов. Даны строки S1 и S2. Напишите программу, которая проверяет, является ли S1 анаграммой S2.
\end{enumerate}

\end{document}
