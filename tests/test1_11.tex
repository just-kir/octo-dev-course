\documentclass[12pt]{article}

\usepackage{listings}
\lstloadlanguages{Ruby}
%\lstset{%
%basicstyle=\ttfamily\color{black},
%commentstyle = \ttfamily\color{red},
%keywordstyle=\ttfamily\color{blue},
%stringstyle=\color{orange}}
%\usepackage[dvipsnames]{xcolor}

\usepackage[utf8]{inputenc}
\usepackage[russian]{babel}
\usepackage{amsmath}
\usepackage{amssymb}
\usepackage{geometry}
\usepackage{graphicx}
\usepackage{braket}
\usepackage{hyperref}
\usepackage{esint}
\geometry{top=2cm} %поле сверху
\geometry{bottom=2cm} %поле снизу
\geometry{left=2cm} %поле справа
\geometry{right=2cm}
\usepackage{graphicx}

%\newcommand{\ud}{\mathop{\mathrm{{}d}}\mathopen{}}

\begin{document}


\section{Задачи}
\subsection{Теория}
\begin{enumerate}
\item Объем сообщения, содержащего $4096$ символов, равен $\frac{1}{512}$ мегабайта. Какова мощность алфавита, с помощью которого записано это сообщение?
\item Вероятность появления символа @ в некотором тексте равна $\frac{1}{8}$. Сколько битов информации несет сообщение о том, что очередной символ текста --- @?
\item В некотором алфавите всего 4 символа: гласные О и А, согласные Ш и Щ. Вероятности их появления в тексте: А --- 0.35; О --- 0.4; Ш --- 0.1; Щ --- 0.15. Сколько битов информации несет сообщение о том, что очередной символ текста --- согласная? 
\item Автобус №25 ходит в два раза чаще, чем автобус №13. Сообщение о том, что к остановке подошел автобус №25 , несет 4 бита информации. Сколько битов информации в сообщении: "К остановке подошел автобус №13"?
\end{enumerate}

\subsection{Практика}
\begin{enumerate}
\item Заданы первый и второй элементы прогрессии. Написать программу, которая вычислит N-ый элемент геометрической и арифметической прогрессий, порожденный данными элементами.
\item Задано одно число в десятичной системе счисления. Требуется написать программу, которая вычислит количество единиц, содержащихся в двоичной записи этого числа.
\item Даны $a$ и $b$. Вычислить НОД($a$, $b$) и НОК($a$, $b$).
\item Дано одно число --- $N$. Нужно посчитать максимальное количество ферзей, которых можно расставить на шахматной доске размером NxN. Задача взята отсюда \url{<http://acmp.ru/?main=task&id_task=86>}, там же можно почитать обсуждение, чтобы понять, как решить задачу.


\end{enumerate}

\end{document}
