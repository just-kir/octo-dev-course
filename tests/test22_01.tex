%%% Поля и разметка страницы %%%
\documentclass[a4paper,12pt]{article}
\usepackage{lscape}		% Для включения альбомных страниц
\pagestyle{empty}		% Не нумеровать
\usepackage[left=2cm,right=2cm,
top=3cm,bottom=1.5cm,bindingoffset=0cm]{geometry} % поля


%%% Кодировки и шрифты %%%
\usepackage{cmap}						% Улучшенный поиск русских слов в полученном pdf-файле
\usepackage[T2A]{fontenc}				% Поддержка русских букв
\usepackage[utf8]{inputenc}				% Кодировка utf8
\usepackage[english, russian]{babel}	% Языки: русский, английский



%%% Математические пакеты %%%
\usepackage{amsthm,amsfonts,amsmath,amssymb,amscd} % Математические дополнения от AMS

%%% Оформление абзацев %%%
\usepackage{indentfirst} % Красная строка

\usepackage{enumitem}
\usepackage{verbatim}
\usepackage{listings}

%%% Цвета %%%
\usepackage[usenames]{color}
\usepackage{color}
\usepackage{colortbl}

%%% Таблицы %%%
\usepackage{longtable}					% Длинные таблицы
\usepackage{multirow,makecell,array}	% Улучшенное форматирование таблиц
\usepackage{multicol}

%%% Общее форматирование
\usepackage[singlelinecheck=off,center]{caption}	% Многострочные подписи
\usepackage{soul}									% Поддержка переносоустойчивых подчёркиваний и зачёркиваний
\usepackage{icomma}

%%% Библиография %%%
\usepackage{cite} % Красивые ссылки на литературу

%%% Гиперссылки %%%
\usepackage[plainpages=false,pdfpagelabels=false]{hyperref}
\definecolor{linkcolor}{rgb}{0.9,0,0}
\definecolor{citecolor}{rgb}{0,0.6,0}
\definecolor{urlcolor}{rgb}{0,0,1}
\hypersetup{
    colorlinks, linkcolor={linkcolor},
    citecolor={citecolor}, urlcolor={urlcolor}
}

%%% Изображения %%%
\usepackage{graphicx}		% Подключаем пакет работы с графикой
\graphicspath{{images/}}	% Пути к изображениям

%%% Выравнивание и переносы %%%
\sloppy					% Избавляемся от переполнений
\clubpenalty=10000		% Запрещаем разрыв страницы после первой строки абзаца
\widowpenalty=10000		% Запрещаем разрыв страницы после последней строки абзаца

%%%Оформление блоков кода %%%
\definecolor{dkgreen}{rgb}{0,0.6,0}
\definecolor{gray}{rgb}{0.5,0.5,0.5}
\definecolor{mauve}{rgb}{0.58,0,0.82}
\lstset{frame=tb,
  language=Java,
  aboveskip=3mm,
  belowskip=3mm,
  showstringspaces=false,
  columns=flexible,
  basicstyle={\small\ttfamily},
  numbers=none,
  numberstyle=\tiny\color{gray},
  keywordstyle=\color{blue},
  commentstyle=\color{dkgreen},
  stringstyle=\color{mauve},
  breaklines=true,
  breakatwhitespace=true,
  tabsize=3
}


%%% Библиография %%%
\makeatletter
\bibliographystyle{utf8gost705u}	% Оформляем библиографию в соответствии с ГОСТ 7.0.5
\renewcommand{\@biblabel}[1]{#1.}	% Заменяем библиографию с квадратных скобок на точку:
\makeatother

%%% Колонтитулы %%%
\let\Sectionmark\sectionmark
\def\sectionmark#1{\def\Sectionname{#1}\Sectionmark{#1}}
\makeatletter
\newcommand*{\currentname}{\@currentlabelname}
\renewcommand{\@oddhead}{\it \vbox{\hbox to \textwidth%
		{\hfil Информатика и ИКТ, 11 класс\hfil\strut}\hbox to \textwidth%
		{\today \hfil Физтех-лицей\strut}\hrule}}
\makeatother




%%%%%%%%%%%%%%%%%%%%%%%%%%%%%%%%%%%%%%%%%%%%%%%%%%%%%%%%%%%%%%%%%%%%%%%%%%%%%%%%%%%
\begin{document}
\begin{center}
\Large{\textbf{Контрольная работа, 22 января}}
\end{center}
\section{Космическая разминка}
Спутник запускается на круговую орбиту Земли таким образом, что его период обращения вкоруг Земли составляет $T$ секунд.
\begin{itemize}
\item Показать, что его высота над поверхностью Земли дается формулой:
$$h =  \left( \frac{\gamma M T^2}{4\pi^2} \right)^{1/3} - R,$$
где $\gamma = 6.67 \cdot 10^{-11} m^3 kg^{-1} s^{-2}$ --- гравитационная постоянная, $M = 5.97 \cdot 10^{24} kg$ --- масса Земли и $R = 6371 km$ --- радиус Земли.
\item Написать функцию, которая принимает в качестве параметра $T$ --- период обращения и выдает в качестве результата высоту, на которую необходимо запустить спутник.
\end{itemize}
\section{Чужой код}
Что делает представленная программа? Подробно напишите алгоритм, по которому она работает. Можете ли вы ускорить программу? Что для этого нужно изменить?
\begin{lstlisting}[language=python]
def fun(n):
  flst = []
  k = 2
  while k <= n:
  	while n%k == 0:
  	  flst.append(k)
  	  n =n / k
  	k +=1
  return flst
\end{lstlisting}
\section{Простой маятник}
В этой задаче вам нужно промоделировать движение простого математического маятника. Пусть $l$ --- длина нити, $g$ --- ускорение свободного падения, а $m$ --- масса предмета. Тогда второй закон Ньютона дает:
$$m \vec{a} = -m \vec{g} \sin \alpha .$$
Считая $\alpha$ малым, легко получается дифференциальное уравнение второго порядка для отклонения маятника от положения равновесия:
$$\ddot{\alpha} + \frac{g}{l} \alpha = 0 .$$
Ваша нужно построить график зависимости угла отклонения от времени. В качестве начальных данные выберите $l$ = 1,  $\alpha_0 = 0.1 \frac{\pi}{4}$ --- начальный угол отклонения, $v_0 = 0$ ---начальная скорость, $t = 4 s$ --- время, в течении которого будет колебаться ваш маятник и шаг $\Delta t = 0.001$. 

\end{document}